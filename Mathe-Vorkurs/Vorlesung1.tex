\documentclass[12pt]{article}
\usepackage[utf8]{inputenc}
\usepackage{amsmath}
\usepackage{amsfonts}
\begin{document}

\section{Mengen und Aussagen}
\subsection{Aussagen}
Aussagen sind objekte, die zwei Bedingungen erfüllen:
\begin{enumerate}
    \item Sie sind Zeichenketten, die in einer Grammatik formuliert sind.
    \item Sie müssen einen eindeutigen Wahrheitsgehalt innehaben (bspw. wahr oder falsch, + oder -).
\end{enumerate}
Aussagen werden mit lateinischen Großbuchstaben abgekürzt. (Großes Alphabet) \\
Die Verknüpfung von Aussagen sieht entsprechend wie folgt aus: \textit{Aus A folgt B} heißt $A\Rightarrow B$.
Die zugehörige Wahrheitstafel: \\ \\
\begin{tabular}{c|c|c}
    A & B & $A\Rightarrow B$ \\
    + & + & + \\
    + & - & - \\
    - & + & + \\
    - & - & +
\end{tabular} 
\\ \\
Analog das Beispiel \textit{A genau dann, wenn B} oder auch \textit{A äquivalent B} heißt $A\Leftrightarrow B$ Die zugehörige Wahrheitstafel: \\ \\
\begin{tabular}{c|c|c}
    A & B & $A\Leftrightarrow B$ \\
    + & + & + \\
    + & - & - \\
    - & + & - \\
    - & - & +
\end{tabular} 
\\ \\
\subsection{Was ist ein Beweis?}
Ein Beweis besteht aus vielen Zwischenaussagen. \\
Beweise:
\begin{eqnarray*}
    A&\Leftrightarrow& B \\
    A\Leftrightarrow A_{1}\Leftrightarrow A_{2}&\Leftrightarrow& ...\Leftrightarrow B
\end{eqnarray*}

\subsection{Quantoren}
\subsubsection{Allquantor}
Der Allquantor $\forall$ drückt aus, dass eine bestimmte Bedingung oder Aussage für alle Objekte zutrifft. Beispiel:
\begin{center}
    Für alle reellen Zahlen x gilt: $(x+1)=x^{2}+2x+5$
\end{center}
\subsubsection{Existenzquantor}
Der Existenzquantor $\exists$ drückt aus, dass es mindestens ein Objekt gibt, welches bestimmte Bedingungen erfüllt. Beispiel:
\begin{center}
    Es gibt ein reelles x mit $x+3=2x+5$
\end{center}

\section{Mengen}
Mengen sind Sammlungen von Objekten. Objekte können in ihr nicht doppelt vorkommen. \\
Die Menge der natürlichen Zahlen $\{0,1,2,3,\dots\}$ wird mit $\mathbb{N}$ abgekürzt. In der Informatik enthält $\mathbb{N}$ immer die Null.
Richtig notiert: 
\begin{eqnarray*}
    \mathbb{N}&:=&\{0,1,2,3,\dots\} \text{ als Menge aller natürlichen Zahlen} \\
    \mathbb{Z}&:=&\{\dots,-2,-1,0,1,2,\dots\} \text{ als Menge aller ganzen Zahlen} \\
\end{eqnarray*}

\end{document}